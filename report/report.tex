\def\year{2018}\relax
%Taken from file: formatting-instruction.tex
\documentclass[letterpaper]{article} %DO NOT CHANGE THIS
% \usepackage{aaai18}  %Required
% \usepackage{times}  %Required
% \usepackage{helvet}  %Required
% \usepackage{courier}  %Required
% \usepackage{url}  %Required
% \usepackage{graphicx}  %Required
% \frenchspacing  %Required
% \setlength{\pdfpagewidth}{8.5in}  %Required
% \setlength{\pdfpageheight}{11in}  %Required
% %PDF Info Is Required:
%   \pdfinfo{
% /Title (Removing Viruses with Machine Learning in Dr.~Mario)
% /Author (Ryan Gately and JJ Brown)}

\author{Ryan Gately and JJ Brown}
\date{\today}
\title{Removing Viruses with Machine Learning in Dr.~Mario}

\begin{document}

\maketitle

\begin{abstract}
We trained machine learning agents that manipulate an emulated NES controller to play Dr.~Mario.
Both Q-learning and SARSA learning algorithms showed some advantage over a random controller.
We compared a SARSA agent that viewed the local ``neighborhood'' around the player's capsule and controlled the controller directly to a Q-learning agent that viewed the entire topmost layer of viruses and executed high-level actions in the game. Ultimately, while we believe the local ``neighborhood'' state space may allow for the highest theoretical quality of performance, the Q-learning agent with a high-level controller yielded the best results, with the fastest rate of learning and the highest average performance.
\end{abstract}

\section{Introduction}

\section{Problem Definition and Approaches}

\section{Experimental Evaluation}

\section{Related Work}

\section{Future Work}

\section{Conclusion}

\section{Bibliography}

\end{document}